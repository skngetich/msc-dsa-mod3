% Options for packages loaded elsewhere
\PassOptionsToPackage{unicode}{hyperref}
\PassOptionsToPackage{hyphens}{url}
%
\documentclass[
]{article}
\title{CAT 2}
\author{79546 - Stephen K. Ng'etich}
\date{}

\usepackage{amsmath,amssymb}
\usepackage{lmodern}
\usepackage{iftex}
\ifPDFTeX
  \usepackage[T1]{fontenc}
  \usepackage[utf8]{inputenc}
  \usepackage{textcomp} % provide euro and other symbols
\else % if luatex or xetex
  \usepackage{unicode-math}
  \defaultfontfeatures{Scale=MatchLowercase}
  \defaultfontfeatures[\rmfamily]{Ligatures=TeX,Scale=1}
\fi
% Use upquote if available, for straight quotes in verbatim environments
\IfFileExists{upquote.sty}{\usepackage{upquote}}{}
\IfFileExists{microtype.sty}{% use microtype if available
  \usepackage[]{microtype}
  \UseMicrotypeSet[protrusion]{basicmath} % disable protrusion for tt fonts
}{}
\makeatletter
\@ifundefined{KOMAClassName}{% if non-KOMA class
  \IfFileExists{parskip.sty}{%
    \usepackage{parskip}
  }{% else
    \setlength{\parindent}{0pt}
    \setlength{\parskip}{6pt plus 2pt minus 1pt}}
}{% if KOMA class
  \KOMAoptions{parskip=half}}
\makeatother
\usepackage{xcolor}
\IfFileExists{xurl.sty}{\usepackage{xurl}}{} % add URL line breaks if available
\IfFileExists{bookmark.sty}{\usepackage{bookmark}}{\usepackage{hyperref}}
\hypersetup{
  pdftitle={CAT 2},
  pdfauthor={79546 - Stephen K. Ng'etich},
  hidelinks,
  pdfcreator={LaTeX via pandoc}}
\urlstyle{same} % disable monospaced font for URLs
\usepackage[margin=1in]{geometry}
\usepackage{color}
\usepackage{fancyvrb}
\newcommand{\VerbBar}{|}
\newcommand{\VERB}{\Verb[commandchars=\\\{\}]}
\DefineVerbatimEnvironment{Highlighting}{Verbatim}{commandchars=\\\{\}}
% Add ',fontsize=\small' for more characters per line
\usepackage{framed}
\definecolor{shadecolor}{RGB}{248,248,248}
\newenvironment{Shaded}{\begin{snugshade}}{\end{snugshade}}
\newcommand{\AlertTok}[1]{\textcolor[rgb]{0.94,0.16,0.16}{#1}}
\newcommand{\AnnotationTok}[1]{\textcolor[rgb]{0.56,0.35,0.01}{\textbf{\textit{#1}}}}
\newcommand{\AttributeTok}[1]{\textcolor[rgb]{0.77,0.63,0.00}{#1}}
\newcommand{\BaseNTok}[1]{\textcolor[rgb]{0.00,0.00,0.81}{#1}}
\newcommand{\BuiltInTok}[1]{#1}
\newcommand{\CharTok}[1]{\textcolor[rgb]{0.31,0.60,0.02}{#1}}
\newcommand{\CommentTok}[1]{\textcolor[rgb]{0.56,0.35,0.01}{\textit{#1}}}
\newcommand{\CommentVarTok}[1]{\textcolor[rgb]{0.56,0.35,0.01}{\textbf{\textit{#1}}}}
\newcommand{\ConstantTok}[1]{\textcolor[rgb]{0.00,0.00,0.00}{#1}}
\newcommand{\ControlFlowTok}[1]{\textcolor[rgb]{0.13,0.29,0.53}{\textbf{#1}}}
\newcommand{\DataTypeTok}[1]{\textcolor[rgb]{0.13,0.29,0.53}{#1}}
\newcommand{\DecValTok}[1]{\textcolor[rgb]{0.00,0.00,0.81}{#1}}
\newcommand{\DocumentationTok}[1]{\textcolor[rgb]{0.56,0.35,0.01}{\textbf{\textit{#1}}}}
\newcommand{\ErrorTok}[1]{\textcolor[rgb]{0.64,0.00,0.00}{\textbf{#1}}}
\newcommand{\ExtensionTok}[1]{#1}
\newcommand{\FloatTok}[1]{\textcolor[rgb]{0.00,0.00,0.81}{#1}}
\newcommand{\FunctionTok}[1]{\textcolor[rgb]{0.00,0.00,0.00}{#1}}
\newcommand{\ImportTok}[1]{#1}
\newcommand{\InformationTok}[1]{\textcolor[rgb]{0.56,0.35,0.01}{\textbf{\textit{#1}}}}
\newcommand{\KeywordTok}[1]{\textcolor[rgb]{0.13,0.29,0.53}{\textbf{#1}}}
\newcommand{\NormalTok}[1]{#1}
\newcommand{\OperatorTok}[1]{\textcolor[rgb]{0.81,0.36,0.00}{\textbf{#1}}}
\newcommand{\OtherTok}[1]{\textcolor[rgb]{0.56,0.35,0.01}{#1}}
\newcommand{\PreprocessorTok}[1]{\textcolor[rgb]{0.56,0.35,0.01}{\textit{#1}}}
\newcommand{\RegionMarkerTok}[1]{#1}
\newcommand{\SpecialCharTok}[1]{\textcolor[rgb]{0.00,0.00,0.00}{#1}}
\newcommand{\SpecialStringTok}[1]{\textcolor[rgb]{0.31,0.60,0.02}{#1}}
\newcommand{\StringTok}[1]{\textcolor[rgb]{0.31,0.60,0.02}{#1}}
\newcommand{\VariableTok}[1]{\textcolor[rgb]{0.00,0.00,0.00}{#1}}
\newcommand{\VerbatimStringTok}[1]{\textcolor[rgb]{0.31,0.60,0.02}{#1}}
\newcommand{\WarningTok}[1]{\textcolor[rgb]{0.56,0.35,0.01}{\textbf{\textit{#1}}}}
\usepackage{longtable,booktabs,array}
\usepackage{calc} % for calculating minipage widths
% Correct order of tables after \paragraph or \subparagraph
\usepackage{etoolbox}
\makeatletter
\patchcmd\longtable{\par}{\if@noskipsec\mbox{}\fi\par}{}{}
\makeatother
% Allow footnotes in longtable head/foot
\IfFileExists{footnotehyper.sty}{\usepackage{footnotehyper}}{\usepackage{footnote}}
\makesavenoteenv{longtable}
\usepackage{graphicx}
\makeatletter
\def\maxwidth{\ifdim\Gin@nat@width>\linewidth\linewidth\else\Gin@nat@width\fi}
\def\maxheight{\ifdim\Gin@nat@height>\textheight\textheight\else\Gin@nat@height\fi}
\makeatother
% Scale images if necessary, so that they will not overflow the page
% margins by default, and it is still possible to overwrite the defaults
% using explicit options in \includegraphics[width, height, ...]{}
\setkeys{Gin}{width=\maxwidth,height=\maxheight,keepaspectratio}
% Set default figure placement to htbp
\makeatletter
\def\fps@figure{htbp}
\makeatother
\setlength{\emergencystretch}{3em} % prevent overfull lines
\providecommand{\tightlist}{%
  \setlength{\itemsep}{0pt}\setlength{\parskip}{0pt}}
\setcounter{secnumdepth}{5}
\ifLuaTeX
  \usepackage{selnolig}  % disable illegal ligatures
\fi

\begin{document}
\maketitle

{
\setcounter{tocdepth}{3}
\tableofcontents
}
\hypertarget{pre-requisite}{%
\section{Pre-requisite}\label{pre-requisite}}

\hypertarget{load-package}{%
\subsection{load package}\label{load-package}}

\begin{Shaded}
\begin{Highlighting}[]
\CommentTok{\# Clear variables}
\FunctionTok{rm}\NormalTok{(}\AttributeTok{list=}\FunctionTok{ls}\NormalTok{())}

\FunctionTok{library}\NormalTok{(readxl)}
\end{Highlighting}
\end{Shaded}

\begin{verbatim}
## Warning: package 'readxl' was built under R version 4.1.3
\end{verbatim}

\begin{Shaded}
\begin{Highlighting}[]
\FunctionTok{library}\NormalTok{(tidyverse)}
\end{Highlighting}
\end{Shaded}

\begin{verbatim}
## Warning: package 'tidyverse' was built under R version 4.1.3
\end{verbatim}

\begin{verbatim}
## -- Attaching packages --------------------------------------- tidyverse 1.3.1 --
\end{verbatim}

\begin{verbatim}
## v ggplot2 3.3.5     v purrr   0.3.4
## v tibble  3.1.3     v dplyr   1.0.8
## v tidyr   1.1.3     v stringr 1.4.0
## v readr   2.0.1     v forcats 0.5.1
\end{verbatim}

\begin{verbatim}
## Warning: package 'ggplot2' was built under R version 4.1.2
\end{verbatim}

\begin{verbatim}
## Warning: package 'dplyr' was built under R version 4.1.3
\end{verbatim}

\begin{verbatim}
## -- Conflicts ------------------------------------------ tidyverse_conflicts() --
## x dplyr::filter() masks stats::filter()
## x dplyr::lag()    masks stats::lag()
\end{verbatim}

\begin{Shaded}
\begin{Highlighting}[]
\FunctionTok{library}\NormalTok{(caret)}
\end{Highlighting}
\end{Shaded}

\begin{verbatim}
## Warning: package 'caret' was built under R version 4.1.3
\end{verbatim}

\begin{verbatim}
## Loading required package: lattice
\end{verbatim}

\begin{verbatim}
## 
## Attaching package: 'caret'
\end{verbatim}

\begin{verbatim}
## The following object is masked from 'package:purrr':
## 
##     lift
\end{verbatim}

\begin{Shaded}
\begin{Highlighting}[]
\FunctionTok{library}\NormalTok{(glmnet)}
\end{Highlighting}
\end{Shaded}

\begin{verbatim}
## Warning: package 'glmnet' was built under R version 4.1.3
\end{verbatim}

\begin{verbatim}
## Loading required package: Matrix
\end{verbatim}

\begin{verbatim}
## 
## Attaching package: 'Matrix'
\end{verbatim}

\begin{verbatim}
## The following objects are masked from 'package:tidyr':
## 
##     expand, pack, unpack
\end{verbatim}

\begin{verbatim}
## Loaded glmnet 4.1-3
\end{verbatim}

\begin{Shaded}
\begin{Highlighting}[]
\DocumentationTok{\#\# set the seed to make your partition reproducible}
\FunctionTok{set.seed}\NormalTok{(}\DecValTok{123}\NormalTok{)}
\end{Highlighting}
\end{Shaded}

\hypertarget{load-dataset}{%
\subsection{Load dataset}\label{load-dataset}}

\begin{Shaded}
\begin{Highlighting}[]
\CommentTok{\# Load Dataset}
\NormalTok{dataset }\OtherTok{\textless{}{-}} \FunctionTok{read\_excel}\NormalTok{(}\StringTok{"dataset/TestData.xlsx"}\NormalTok{)}
\end{Highlighting}
\end{Shaded}

\hypertarget{question}{%
\section{Question}\label{question}}

\hypertarget{a-split-the-data-set-into-75-training-set-and-25-test-set.}{%
\subsection{(a) Split the data set into 75\% training set and 25\% test
set.}\label{a-split-the-data-set-into-75-training-set-and-25-test-set.}}

\begin{Shaded}
\begin{Highlighting}[]
\DocumentationTok{\#\# 75\% of the sample size}
\NormalTok{sample\_size }\OtherTok{\textless{}{-}} \FunctionTok{floor}\NormalTok{(}\FloatTok{0.75} \SpecialCharTok{*} \FunctionTok{nrow}\NormalTok{(dataset))}


\NormalTok{train\_ind }\OtherTok{\textless{}{-}} \FunctionTok{sample}\NormalTok{(}\FunctionTok{seq\_len}\NormalTok{(}\FunctionTok{nrow}\NormalTok{(dataset)), }\AttributeTok{size =}\NormalTok{ sample\_size)}

\NormalTok{train }\OtherTok{\textless{}{-}}\NormalTok{ dataset[train\_ind, ]}
\NormalTok{test }\OtherTok{\textless{}{-}}\NormalTok{ dataset[}\SpecialCharTok{{-}}\NormalTok{train\_ind, ]}
\end{Highlighting}
\end{Shaded}

\hypertarget{b-fit-a-linear-model-using-least-squares-on-the-training-set-and-report-the-test-error-obtained.}{%
\subsection{(b) Fit a linear model using least squares on the training
set, and report the test error
obtained.}\label{b-fit-a-linear-model-using-least-squares-on-the-training-set-and-report-the-test-error-obtained.}}

\begin{Shaded}
\begin{Highlighting}[]
\NormalTok{lm\_model }\OtherTok{=} \FunctionTok{lm}\NormalTok{(Response }\SpecialCharTok{\textasciitilde{}}\NormalTok{ . , }\AttributeTok{data=}\NormalTok{train)}
\CommentTok{\#summary(lm\_model)}

\NormalTok{predictions }\OtherTok{=} \FunctionTok{predict.lm}\NormalTok{(lm\_model,}\AttributeTok{newdata =}\NormalTok{ test)}

\CommentTok{\#Model performance metrics}
\NormalTok{ml\_performance.lse}\OtherTok{=}\FunctionTok{data.frame}\NormalTok{( }
            \AttributeTok{MODEL =} \StringTok{"Least Squares"}\NormalTok{,}
            \AttributeTok{s1 =} \FunctionTok{R2}\NormalTok{(predictions, test}\SpecialCharTok{$}\NormalTok{Response),}
            \AttributeTok{RMSE =} \FunctionTok{RMSE}\NormalTok{(predictions, test}\SpecialCharTok{$}\NormalTok{Response),}
            \AttributeTok{MAE =} \FunctionTok{MAE}\NormalTok{(predictions, test}\SpecialCharTok{$}\NormalTok{Response))}

\NormalTok{ml\_performance.lse}
\end{Highlighting}
\end{Shaded}

\begin{verbatim}
##           MODEL        s1     RMSE      MAE
## 1 Least Squares 0.9908415 9.955736 8.279993
\end{verbatim}

\hypertarget{c-fit-a-ridge-regression-model-on-the-training-set-with-ux3bb-chosen-by-cross-validation.-report-the-test-error-obtained.}{%
\subsection{(c) Fit a ridge regression model on the training set, with λ
chosen by cross-validation. Report the test error
obtained.}\label{c-fit-a-ridge-regression-model-on-the-training-set-with-ux3bb-chosen-by-cross-validation.-report-the-test-error-obtained.}}

\begin{Shaded}
\begin{Highlighting}[]
\NormalTok{train.matrix }\OtherTok{=} \FunctionTok{model.matrix}\NormalTok{(Response}\SpecialCharTok{\textasciitilde{}}\NormalTok{., }\AttributeTok{data =}\NormalTok{ train)}
\NormalTok{test.matrix }\OtherTok{=} \FunctionTok{model.matrix}\NormalTok{(Response}\SpecialCharTok{\textasciitilde{}}\NormalTok{., }\AttributeTok{data =}\NormalTok{ test)}

\CommentTok{\#Choose lambda using cross{-}validation}
\NormalTok{crossvalidation }\OtherTok{=} \FunctionTok{cv.glmnet}\NormalTok{(train.matrix,train}\SpecialCharTok{$}\NormalTok{Response,}\AttributeTok{alpha=}\DecValTok{0}\NormalTok{)}
\FunctionTok{plot}\NormalTok{(crossvalidation)}
\end{Highlighting}
\end{Shaded}

\includegraphics{CAT1_files/figure-latex/unnamed-chunk-5-1.pdf}

\begin{Shaded}
\begin{Highlighting}[]
\NormalTok{bestlamda }\OtherTok{=}\NormalTok{ crossvalidation}\SpecialCharTok{$}\NormalTok{lambda.min}
\NormalTok{bestlamda}
\end{Highlighting}
\end{Shaded}

\begin{verbatim}
## [1] 10.24674
\end{verbatim}

\begin{Shaded}
\begin{Highlighting}[]
\CommentTok{\#Fit a ridge regression}
\NormalTok{ridge\_model }\OtherTok{=} \FunctionTok{glmnet}\NormalTok{(train.matrix,train}\SpecialCharTok{$}\NormalTok{Response,}\AttributeTok{alpha =} \DecValTok{0}\NormalTok{)}
\CommentTok{\#Make predictions}
\NormalTok{ridge\_predictions }\OtherTok{=} \FunctionTok{predict}\NormalTok{(ridge\_model,}\AttributeTok{s=}\NormalTok{bestlamda,}\AttributeTok{newx =}\NormalTok{ test.matrix)}
\CommentTok{\#Calculate test error}

\CommentTok{\#Model performance metrics}
\NormalTok{ml\_performance.ridge }\OtherTok{=} \FunctionTok{data.frame}\NormalTok{( }
                        \AttributeTok{MODEL =} \StringTok{"Ridge regression"}\NormalTok{,}
            \AttributeTok{R2 =} \FunctionTok{R2}\NormalTok{(ridge\_predictions, test}\SpecialCharTok{$}\NormalTok{Response),}
            \AttributeTok{RMSE =} \FunctionTok{RMSE}\NormalTok{(ridge\_predictions, test}\SpecialCharTok{$}\NormalTok{Response),}
            \AttributeTok{MAE =} \FunctionTok{MAE}\NormalTok{(ridge\_predictions, test}\SpecialCharTok{$}\NormalTok{Response))}
\NormalTok{ml\_performance.ridge}
\end{Highlighting}
\end{Shaded}

\begin{verbatim}
##              MODEL        s1     RMSE      MAE
## 1 Ridge regression 0.9809305 14.99081 11.34165
\end{verbatim}

\#\#(d) Fit a lasso model on the training set, with λ chosen by cross
validation. Report the test error obtained, along with the number of
non-zero coefficient estimates.

\begin{Shaded}
\begin{Highlighting}[]
\NormalTok{train.matrix }\OtherTok{=} \FunctionTok{model.matrix}\NormalTok{(Response}\SpecialCharTok{\textasciitilde{}}\NormalTok{., }\AttributeTok{data =}\NormalTok{ train)}
\NormalTok{test.matrix }\OtherTok{=} \FunctionTok{model.matrix}\NormalTok{(Response}\SpecialCharTok{\textasciitilde{}}\NormalTok{., }\AttributeTok{data =}\NormalTok{ test)}

\CommentTok{\#Choose lambda using cross{-}validation}
\NormalTok{crossvalidation1 }\OtherTok{=} \FunctionTok{cv.glmnet}\NormalTok{(train.matrix,train}\SpecialCharTok{$}\NormalTok{Response,}\AttributeTok{alpha=}\DecValTok{1}\NormalTok{)}
\FunctionTok{plot}\NormalTok{(crossvalidation)}
\end{Highlighting}
\end{Shaded}

\includegraphics{CAT1_files/figure-latex/unnamed-chunk-7-1.pdf}

\begin{Shaded}
\begin{Highlighting}[]
\NormalTok{bestlamda1 }\OtherTok{=}\NormalTok{ crossvalidation1}\SpecialCharTok{$}\NormalTok{lambda.min}
\NormalTok{bestlamda1}
\end{Highlighting}
\end{Shaded}

\begin{verbatim}
## [1] 0.5597055
\end{verbatim}

\begin{Shaded}
\begin{Highlighting}[]
\CommentTok{\#Fit lasso model}
\CommentTok{\# Note alpha=1 for lasso only and can blend with ridge penalty down to}
\NormalTok{lasso\_model }\OtherTok{=} \FunctionTok{glmnet}\NormalTok{(train.matrix,train}\SpecialCharTok{$}\NormalTok{Response,}\AttributeTok{alpha=}\DecValTok{1}\NormalTok{)}
\CommentTok{\#Make predictions}
\NormalTok{lasso\_predictions }\OtherTok{=} \FunctionTok{predict}\NormalTok{(lasso\_model,}\AttributeTok{s=}\NormalTok{bestlamda1,}\AttributeTok{newx=}\NormalTok{test.matrix)}
\CommentTok{\#Model performance metrics}
\NormalTok{ml\_performance.lasso }\OtherTok{=} \FunctionTok{data.frame}\NormalTok{(}
  \AttributeTok{MODEL =} \StringTok{"Lasso regression"}\NormalTok{,}
  \AttributeTok{R2 =} \FunctionTok{R2}\NormalTok{(lasso\_predictions, test}\SpecialCharTok{$}\NormalTok{Response),}
            \AttributeTok{RMSE =} \FunctionTok{RMSE}\NormalTok{(lasso\_predictions, test}\SpecialCharTok{$}\NormalTok{Response),}
            \AttributeTok{MAE =} \FunctionTok{MAE}\NormalTok{(lasso\_predictions, test}\SpecialCharTok{$}\NormalTok{Response))}
\NormalTok{ml\_performance.lasso}
\end{Highlighting}
\end{Shaded}

\begin{verbatim}
##              MODEL        s1     RMSE      MAE
## 1 Lasso regression 0.9901851 10.33082 8.359939
\end{verbatim}

\hypertarget{e-comment-on-the-results-obtained.-how-accurately-can-we-predict-the-response-variable-is-there-much-difference-among-the-test-errors-resulting-from-these-three-approaches-present-and-discuss-results-for-the-approaches}{%
\subsection{(e) Comment on the results obtained. How accurately can we
predict the response variable? Is there much difference among the test
errors resulting from these three approaches? Present and discuss
results for the
approaches}\label{e-comment-on-the-results-obtained.-how-accurately-can-we-predict-the-response-variable-is-there-much-difference-among-the-test-errors-resulting-from-these-three-approaches-present-and-discuss-results-for-the-approaches}}

The following table represents model performance

\begin{Shaded}
\begin{Highlighting}[]
\NormalTok{t }\OtherTok{=} \FunctionTok{rbind}\NormalTok{(ml\_performance.lse,ml\_performance.lasso,ml\_performance.ridge)}
\NormalTok{knitr}\SpecialCharTok{::}\FunctionTok{kable}\NormalTok{(t,}\StringTok{"simple"}\NormalTok{)}
\end{Highlighting}
\end{Shaded}

\begin{longtable}[]{@{}lrrr@{}}
\toprule
MODEL & s1 & RMSE & MAE \\
\midrule
\endhead
Least Squares & 0.9908415 & 9.955736 & 8.279993 \\
Lasso regression & 0.9901851 & 10.330821 & 8.359939 \\
Ridge regression & 0.9809305 & 14.990815 & 11.341649 \\
The best model in p & redicting th & e responce v & ariable is
\texttt{Least\ Squares} since it has the least Root Mean Square Error
while Ridge regression has the worst model perfomance. \\
\bottomrule
\end{longtable}

\texttt{Ridge\ Regression} has the highest margin in test errors compare
to the other 2 regression models

\end{document}
