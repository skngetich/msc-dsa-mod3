% Options for packages loaded elsewhere
\PassOptionsToPackage{unicode}{hyperref}
\PassOptionsToPackage{hyphens}{url}
%
\documentclass[
]{article}
\title{assignment-5}
\author{Stephen K. Ng'etich}
\date{}

\usepackage{amsmath,amssymb}
\usepackage{lmodern}
\usepackage{iftex}
\ifPDFTeX
  \usepackage[T1]{fontenc}
  \usepackage[utf8]{inputenc}
  \usepackage{textcomp} % provide euro and other symbols
\else % if luatex or xetex
  \usepackage{unicode-math}
  \defaultfontfeatures{Scale=MatchLowercase}
  \defaultfontfeatures[\rmfamily]{Ligatures=TeX,Scale=1}
\fi
% Use upquote if available, for straight quotes in verbatim environments
\IfFileExists{upquote.sty}{\usepackage{upquote}}{}
\IfFileExists{microtype.sty}{% use microtype if available
  \usepackage[]{microtype}
  \UseMicrotypeSet[protrusion]{basicmath} % disable protrusion for tt fonts
}{}
\makeatletter
\@ifundefined{KOMAClassName}{% if non-KOMA class
  \IfFileExists{parskip.sty}{%
    \usepackage{parskip}
  }{% else
    \setlength{\parindent}{0pt}
    \setlength{\parskip}{6pt plus 2pt minus 1pt}}
}{% if KOMA class
  \KOMAoptions{parskip=half}}
\makeatother
\usepackage{xcolor}
\IfFileExists{xurl.sty}{\usepackage{xurl}}{} % add URL line breaks if available
\IfFileExists{bookmark.sty}{\usepackage{bookmark}}{\usepackage{hyperref}}
\hypersetup{
  pdftitle={assignment-5},
  pdfauthor={Stephen K. Ng'etich},
  hidelinks,
  pdfcreator={LaTeX via pandoc}}
\urlstyle{same} % disable monospaced font for URLs
\usepackage[margin=1in]{geometry}
\usepackage{color}
\usepackage{fancyvrb}
\newcommand{\VerbBar}{|}
\newcommand{\VERB}{\Verb[commandchars=\\\{\}]}
\DefineVerbatimEnvironment{Highlighting}{Verbatim}{commandchars=\\\{\}}
% Add ',fontsize=\small' for more characters per line
\usepackage{framed}
\definecolor{shadecolor}{RGB}{248,248,248}
\newenvironment{Shaded}{\begin{snugshade}}{\end{snugshade}}
\newcommand{\AlertTok}[1]{\textcolor[rgb]{0.94,0.16,0.16}{#1}}
\newcommand{\AnnotationTok}[1]{\textcolor[rgb]{0.56,0.35,0.01}{\textbf{\textit{#1}}}}
\newcommand{\AttributeTok}[1]{\textcolor[rgb]{0.77,0.63,0.00}{#1}}
\newcommand{\BaseNTok}[1]{\textcolor[rgb]{0.00,0.00,0.81}{#1}}
\newcommand{\BuiltInTok}[1]{#1}
\newcommand{\CharTok}[1]{\textcolor[rgb]{0.31,0.60,0.02}{#1}}
\newcommand{\CommentTok}[1]{\textcolor[rgb]{0.56,0.35,0.01}{\textit{#1}}}
\newcommand{\CommentVarTok}[1]{\textcolor[rgb]{0.56,0.35,0.01}{\textbf{\textit{#1}}}}
\newcommand{\ConstantTok}[1]{\textcolor[rgb]{0.00,0.00,0.00}{#1}}
\newcommand{\ControlFlowTok}[1]{\textcolor[rgb]{0.13,0.29,0.53}{\textbf{#1}}}
\newcommand{\DataTypeTok}[1]{\textcolor[rgb]{0.13,0.29,0.53}{#1}}
\newcommand{\DecValTok}[1]{\textcolor[rgb]{0.00,0.00,0.81}{#1}}
\newcommand{\DocumentationTok}[1]{\textcolor[rgb]{0.56,0.35,0.01}{\textbf{\textit{#1}}}}
\newcommand{\ErrorTok}[1]{\textcolor[rgb]{0.64,0.00,0.00}{\textbf{#1}}}
\newcommand{\ExtensionTok}[1]{#1}
\newcommand{\FloatTok}[1]{\textcolor[rgb]{0.00,0.00,0.81}{#1}}
\newcommand{\FunctionTok}[1]{\textcolor[rgb]{0.00,0.00,0.00}{#1}}
\newcommand{\ImportTok}[1]{#1}
\newcommand{\InformationTok}[1]{\textcolor[rgb]{0.56,0.35,0.01}{\textbf{\textit{#1}}}}
\newcommand{\KeywordTok}[1]{\textcolor[rgb]{0.13,0.29,0.53}{\textbf{#1}}}
\newcommand{\NormalTok}[1]{#1}
\newcommand{\OperatorTok}[1]{\textcolor[rgb]{0.81,0.36,0.00}{\textbf{#1}}}
\newcommand{\OtherTok}[1]{\textcolor[rgb]{0.56,0.35,0.01}{#1}}
\newcommand{\PreprocessorTok}[1]{\textcolor[rgb]{0.56,0.35,0.01}{\textit{#1}}}
\newcommand{\RegionMarkerTok}[1]{#1}
\newcommand{\SpecialCharTok}[1]{\textcolor[rgb]{0.00,0.00,0.00}{#1}}
\newcommand{\SpecialStringTok}[1]{\textcolor[rgb]{0.31,0.60,0.02}{#1}}
\newcommand{\StringTok}[1]{\textcolor[rgb]{0.31,0.60,0.02}{#1}}
\newcommand{\VariableTok}[1]{\textcolor[rgb]{0.00,0.00,0.00}{#1}}
\newcommand{\VerbatimStringTok}[1]{\textcolor[rgb]{0.31,0.60,0.02}{#1}}
\newcommand{\WarningTok}[1]{\textcolor[rgb]{0.56,0.35,0.01}{\textbf{\textit{#1}}}}
\usepackage{graphicx}
\makeatletter
\def\maxwidth{\ifdim\Gin@nat@width>\linewidth\linewidth\else\Gin@nat@width\fi}
\def\maxheight{\ifdim\Gin@nat@height>\textheight\textheight\else\Gin@nat@height\fi}
\makeatother
% Scale images if necessary, so that they will not overflow the page
% margins by default, and it is still possible to overwrite the defaults
% using explicit options in \includegraphics[width, height, ...]{}
\setkeys{Gin}{width=\maxwidth,height=\maxheight,keepaspectratio}
% Set default figure placement to htbp
\makeatletter
\def\fps@figure{htbp}
\makeatother
\setlength{\emergencystretch}{3em} % prevent overfull lines
\providecommand{\tightlist}{%
  \setlength{\itemsep}{0pt}\setlength{\parskip}{0pt}}
\setcounter{secnumdepth}{-\maxdimen} % remove section numbering
\ifLuaTeX
  \usepackage{selnolig}  % disable illegal ligatures
\fi

\begin{document}
\maketitle

\hypertarget{pre-requisite}{%
\section{Pre-requisite}\label{pre-requisite}}

\hypertarget{load-packages}{%
\subsection{Load Packages}\label{load-packages}}

\begin{Shaded}
\begin{Highlighting}[]
\CommentTok{\# Clear variables}
\FunctionTok{rm}\NormalTok{(}\AttributeTok{list=}\FunctionTok{ls}\NormalTok{())}

\FunctionTok{library}\NormalTok{(glmnet)}
\FunctionTok{library}\NormalTok{(dplyr)}
\FunctionTok{library}\NormalTok{(ggplot2)}
\FunctionTok{library}\NormalTok{(gam)}
\FunctionTok{library}\NormalTok{(readxl)}
\end{Highlighting}
\end{Shaded}

\hypertarget{load-dataset}{%
\subsection{Load Dataset}\label{load-dataset}}

\begin{Shaded}
\begin{Highlighting}[]
\FunctionTok{set.seed}\NormalTok{(}\DecValTok{475}\NormalTok{)                              }

\NormalTok{dataset }\OtherTok{\textless{}{-}} \FunctionTok{read\_excel}\NormalTok{(}\StringTok{"dataset/Dataset7.xlsx"}\NormalTok{)}
\end{Highlighting}
\end{Shaded}

\hypertarget{question}{%
\section{Question}\label{question}}

\hypertarget{split-the-data-set-into-a-training-set-and-a-test-set.}{%
\subsection{Split the data set into a training set and a test
set.}\label{split-the-data-set-into-a-training-set-and-a-test-set.}}

The dataset is split into training and test set in the ratio of 7:3

\begin{Shaded}
\begin{Highlighting}[]
\NormalTok{index }\OtherTok{\textless{}{-}} \FunctionTok{sample}\NormalTok{(}\AttributeTok{x=}\FunctionTok{nrow}\NormalTok{(dataset), }\AttributeTok{size=}\NormalTok{.}\DecValTok{70}\SpecialCharTok{*}\FunctionTok{nrow}\NormalTok{(dataset))}
\NormalTok{train }\OtherTok{\textless{}{-}}\NormalTok{ dataset[index,]}
\NormalTok{test }\OtherTok{\textless{}{-}}\NormalTok{  dataset[}\SpecialCharTok{{-}}\NormalTok{index,]}
\end{Highlighting}
\end{Shaded}

\hypertarget{fit-a-linear-model-using-least-squares-on-the-training-set-and-report-the-test-error-obtained.}{%
\subsection{Fit a linear model using least squares on the training set,
and report the test error
obtained.}\label{fit-a-linear-model-using-least-squares-on-the-training-set-and-report-the-test-error-obtained.}}

\hypertarget{fit-the-model}{%
\subsubsection{fit the model}\label{fit-the-model}}

\begin{Shaded}
\begin{Highlighting}[]
\CommentTok{\# fit the regression model}
\NormalTok{lm\_model }\OtherTok{=} \FunctionTok{lm}\NormalTok{(Profit }\SpecialCharTok{\textasciitilde{}}\NormalTok{ ., }\AttributeTok{data =}\NormalTok{ train)}

\CommentTok{\# get model summary}
\NormalTok{lm\_model\_summary }\OtherTok{=} \FunctionTok{summary}\NormalTok{(lm\_model)}

\FunctionTok{print}\NormalTok{(lm\_model\_summary)}
\end{Highlighting}
\end{Shaded}

\begin{verbatim}
## 
## Call:
## lm(formula = Profit ~ ., data = train)
## 
## Residuals:
##     Min      1Q  Median      3Q     Max 
## -10.617  -3.288  -0.218   2.960  88.830 
## 
## Coefficients:
##              Estimate Std. Error t value Pr(>|t|)    
## (Intercept) 2764.9589   110.7827  24.958  < 2e-16 ***
## Expenses      -4.8434     0.7798  -6.211 8.14e-10 ***
## Adverts        6.1042     0.5563  10.974  < 2e-16 ***
## System         0.7306     0.4493   1.626    0.104    
## Furniture     13.5361     0.2417  56.001  < 2e-16 ***
## Remittance     0.8179     0.6099   1.341    0.180    
## Debts          0.2867     0.1959   1.464    0.144    
## ---
## Signif. codes:  0 '***' 0.001 '**' 0.01 '*' 0.05 '.' 0.1 ' ' 1
## 
## Residual standard error: 5.281 on 867 degrees of freedom
## Multiple R-squared:  0.9998, Adjusted R-squared:  0.9998 
## F-statistic: 9.432e+05 on 6 and 867 DF,  p-value: < 2.2e-16
\end{verbatim}

From the fitted regression model
\texttt{Expenses},\texttt{Adverts},\texttt{system} and
\texttt{Furniture} are significant predictors of Profit at 95\%
confidence interval.The estimated model has an adjusted error of 99.9\%.

The linear regression can be summarized as:
\[Profit = 2729.4862 - 5.3853\,\text{Expenses} + 6.2612\,\text{Adverts}+ 0.9028\,\text{System}+ 13.6387\,\text{Furniture}\]
\#\#\# Calculate the Mean Squeared Error

\begin{Shaded}
\begin{Highlighting}[]
\NormalTok{lm\_model\_pred }\OtherTok{\textless{}{-}} \FunctionTok{predict}\NormalTok{(lm\_model, test)}

\NormalTok{lm\_model\_mse }\OtherTok{\textless{}{-}} \FunctionTok{mean}\NormalTok{((lm\_model\_pred }\SpecialCharTok{{-}}\NormalTok{ test}\SpecialCharTok{$}\NormalTok{Profit)}\SpecialCharTok{\^{}}\DecValTok{2}\NormalTok{)}

\FunctionTok{cat}\NormalTok{(}\StringTok{"The Mean Square Error for the linear regression :"}\NormalTok{ ,lm\_model\_mse)}
\end{Highlighting}
\end{Shaded}

\begin{verbatim}
## The Mean Square Error for the linear regression : 21.7742
\end{verbatim}

\hypertarget{fit-a-ridge-regression-model-on-the-training-set-with-ux3bb-chosen-by-cross-validation.report-the-test-error-obtained.}{%
\subsection{\texorpdfstring{Fit a ridge regression model on the training
set, with \(λ\) chosen by cross-validation.Report the test error
obtained.}{Fit a ridge regression model on the training set, with λ chosen by cross-validation.Report the test error obtained.}}\label{fit-a-ridge-regression-model-on-the-training-set-with-ux3bb-chosen-by-cross-validation.report-the-test-error-obtained.}}

\begin{Shaded}
\begin{Highlighting}[]
\CommentTok{\#All values of x without the profit}
\NormalTok{x\_train }\OtherTok{=} \FunctionTok{data.matrix}\NormalTok{(train[}\SpecialCharTok{{-}}\DecValTok{1}\NormalTok{])}

\CommentTok{\#Values of Y only}
\NormalTok{y\_train }\OtherTok{=}\NormalTok{ train}\SpecialCharTok{$}\NormalTok{Profit}

\CommentTok{\#Find the optimal lambda value via cross validation}
\NormalTok{cv.out}\OtherTok{=}\FunctionTok{cv.glmnet}\NormalTok{(x\_train,y\_train,}\AttributeTok{alpha=}\DecValTok{0}\NormalTok{)}
\NormalTok{bestlam}\OtherTok{=}\NormalTok{cv.out}\SpecialCharTok{$}\NormalTok{lambda.min}

\FunctionTok{cat}\NormalTok{(}\StringTok{"Optimal lambda value for cross validation"}\NormalTok{,bestlam, }\StringTok{"  }\SpecialCharTok{\textbackslash{}n}\StringTok{"}\NormalTok{)}
\end{Highlighting}
\end{Shaded}

\begin{verbatim}
## Optimal lambda value for cross validation 42.49277
\end{verbatim}

\begin{Shaded}
\begin{Highlighting}[]
\CommentTok{\#Define lambda grid to be used through out analysis}
\NormalTok{grid}\OtherTok{=}\DecValTok{10}\SpecialCharTok{\^{}}\FunctionTok{seq}\NormalTok{(}\DecValTok{10}\NormalTok{,}\SpecialCharTok{{-}}\DecValTok{2}\NormalTok{,}\AttributeTok{length=}\DecValTok{100}\NormalTok{)}

\CommentTok{\#Fit a ridge regression model}
\NormalTok{ridge.mod}\OtherTok{=}\FunctionTok{glmnet}\NormalTok{(x\_train,y\_train,}\AttributeTok{alpha =} \DecValTok{0}\NormalTok{, }\AttributeTok{lambda=}\NormalTok{grid)}

\NormalTok{x\_test }\OtherTok{=} \FunctionTok{data.matrix}\NormalTok{(test[}\SpecialCharTok{{-}}\DecValTok{1}\NormalTok{])}
\NormalTok{y\_test }\OtherTok{=}\NormalTok{ test}\SpecialCharTok{$}\NormalTok{Profit}

\CommentTok{\#Compute the test error w/ lambda chosen by cross validation}
\NormalTok{ridge.pred}\OtherTok{=}\FunctionTok{predict}\NormalTok{(ridge.mod,}\AttributeTok{s=}\NormalTok{bestlam,}\AttributeTok{newx=}\NormalTok{x\_test)}
\NormalTok{ridge.mse}\OtherTok{=}\FunctionTok{round}\NormalTok{(}\FunctionTok{mean}\NormalTok{((ridge.pred}\SpecialCharTok{{-}}\NormalTok{y\_test)}\SpecialCharTok{\^{}}\DecValTok{2}\NormalTok{))}
\FunctionTok{cat}\NormalTok{(}\StringTok{"The Mean Square Error for the linear regression :"}\NormalTok{ ,ridge.mse)}
\end{Highlighting}
\end{Shaded}

\begin{verbatim}
## The Mean Square Error for the linear regression : 439
\end{verbatim}

\begin{Shaded}
\begin{Highlighting}[]
\CommentTok{\#Store ridge coefficients}
\NormalTok{ridge.coef}\OtherTok{=}\FunctionTok{predict}\NormalTok{(ridge.mod,}\AttributeTok{type=}\StringTok{"coefficients"}\NormalTok{,}\AttributeTok{s=}\NormalTok{bestlam)}
\end{Highlighting}
\end{Shaded}

\hypertarget{fit-a-lasso-model-on-the-training-set-with-ux3bb-chosen-by-crossvalidation.-report-the-test-error-obtained-along-with-the-number-of-non-zero-coefficient-estimates.}{%
\subsection{\texorpdfstring{Fit a lasso model on the training set, with
\(λ\) chosen by crossvalidation. Report the test error obtained, along
with the number of non-zero coefficient
estimates.}{Fit a lasso model on the training set, with λ chosen by crossvalidation. Report the test error obtained, along with the number of non-zero coefficient estimates.}}\label{fit-a-lasso-model-on-the-training-set-with-ux3bb-chosen-by-crossvalidation.-report-the-test-error-obtained-along-with-the-number-of-non-zero-coefficient-estimates.}}

\begin{Shaded}
\begin{Highlighting}[]
\CommentTok{\#Find the optimal lambda value via cross validation}
\NormalTok{cv.out}\OtherTok{=}\FunctionTok{cv.glmnet}\NormalTok{(x\_train,y\_train,}\AttributeTok{alpha=}\DecValTok{1}\NormalTok{)}
\NormalTok{bestlam}\OtherTok{=}\NormalTok{cv.out}\SpecialCharTok{$}\NormalTok{lambda.min}
\FunctionTok{cat}\NormalTok{(}\StringTok{"Optimal lambda value for cross validation"}\NormalTok{,bestlam, }\StringTok{"  }\SpecialCharTok{\textbackslash{}n}\StringTok{"}\NormalTok{)}
\end{Highlighting}
\end{Shaded}

\begin{verbatim}
## Optimal lambda value for cross validation 11.28645
\end{verbatim}

\begin{Shaded}
\begin{Highlighting}[]
\CommentTok{\#Train the model}
\NormalTok{lasso.mod}\OtherTok{=}\FunctionTok{glmnet}\NormalTok{(x\_train,y\_train,}\AttributeTok{alpha =} \DecValTok{1}\NormalTok{, }\AttributeTok{lambda=}\NormalTok{grid)}

\CommentTok{\#Compute the test error}
\NormalTok{lasso.pred}\OtherTok{=}\FunctionTok{predict}\NormalTok{(lasso.mod,}\AttributeTok{s=}\NormalTok{bestlam,}\AttributeTok{newx=}\NormalTok{x\_test)}
\NormalTok{lasso.mse}\OtherTok{=}\FunctionTok{round}\NormalTok{(}\FunctionTok{mean}\NormalTok{((lasso.pred}\SpecialCharTok{{-}}\NormalTok{y\_test)}\SpecialCharTok{\^{}}\DecValTok{2}\NormalTok{))}
\CommentTok{\#Store lasso coefficients}
\NormalTok{lasso.coef}\OtherTok{=}\FunctionTok{predict}\NormalTok{(lasso.mod,}\AttributeTok{type=}\StringTok{"coefficients"}\NormalTok{,}\AttributeTok{s=}\NormalTok{bestlam)}
\FunctionTok{cat}\NormalTok{(}\StringTok{"The Mean Square Error for the linear regression :"}\NormalTok{ ,ridge.mse)}
\end{Highlighting}
\end{Shaded}

\begin{verbatim}
## The Mean Square Error for the linear regression : 439
\end{verbatim}

\begin{enumerate}
\def\labelenumi{\arabic{enumi}.}
\setcounter{enumi}{4}
\tightlist
\item
  Fit a PCR model on the training set, with M chosen by crossvalidation.
  Report the test error obtained, along with the value of M selected by
  cross-validation.\\
\item
  Fit a PLS model on the training set, with M chosen by crossvalidation.
  Report the test error obtained, along with the value of M selected by
  cross-validation.\\
\item
  Comment on the results obtained. How accurately can we predict the
  number of college applications received? Is there much difference
  among the test errors resulting from these five approaches?
\end{enumerate}

\end{document}
